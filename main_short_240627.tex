\documentclass[12pt]{article}
\usepackage[utf8]{inputenc}
\usepackage{amsmath,setspace,geometry}
\usepackage{amsthm}
\usepackage{amsfonts}
\usepackage[shortlabels]{enumitem}
\usepackage{rotating}
\usepackage{pdflscape}
\usepackage{graphicx}
\usepackage{bbm}
\usepackage[dvipsnames]{xcolor}
\usepackage{hyperref}
\hypersetup{colorlinks=true, linkcolor= BrickRed, citecolor = BrickRed, filecolor = BrickRed, urlcolor = BrickRed, hypertexnames = true}
\usepackage[]{natbib} 
\bibpunct[:]{(}{)}{,}{a}{}{,}
\geometry{left = 1.0in,right = 1.0in,top = 1.0in,bottom = 1.0in}
\usepackage[english]{babel}
\usepackage{float}
\usepackage{caption}
\usepackage{subcaption}
\usepackage{booktabs}
\usepackage{pdfpages}
\usepackage{threeparttable}
\usepackage{lscape}
\usepackage{bm}
\usepackage{multirow,bigdelim}
%\usepackage[top=15truemm]{geometry}
%\usepackage[]{natbib} 
\bibpunct[:]{(}{)}{,}{a}{}{,}
\setlength{\textwidth}{\paperwidth}     % ひとまず紙面を本文領域に
\setlength{\oddsidemargin}{-5.4truemm}  % 左の余白を20mm(=1inch-5.4mm)に
\setlength{\evensidemargin}{-5.4truemm} % 
\addtolength{\textwidth}{-40truemm}     % 右の余白も20mmに
\renewcommand{\baselinestretch}{0.3}
\newtheorem{proposition}{Proposition}

\setcounter{MaxMatrixCols}{20}

\usepackage{setspace}
\setstretch{1.2}
\begin{document}
\title{A Note on Identification of Match Fixed Effects as Interpretable Unobserved Match Affinity}
\author{Suguru Otani\thanks{suguru.otani@e.u-tokyo.ac.jp, Market Design Center, University of Tokyo}, Tohya Sugano\thanks{sugano-tohya1011@g.ecc.u-tokyo.ac.jp, University of Tokyo\\Declarations of interest: none} }
\maketitle

\begin{abstract}
\noindent
%150 words:
We highlight that match fixed effects, represented by the coefficients of interaction terms involving dummy variables for two elements, lack identification without specific restrictions on parameters. Consequently, the coefficients typically reported as relative match fixed effects by statistical software are not interpretable. To address this, we establish normalization conditions that enable identification of match fixed effect parameters as interpretable indicators of unobserved match affinity, facilitating comparisons among observed matches.
%\textcolor{blue}{Using data from middle school students in the 2007 Trends in International Mathematics and Science Study (TIMSS), we present the distribution of comparable match fixed effects.}
\\
%100 words AER
\textbf{Keywords}: Match fixed effect, Affinity, Identification \\
\textbf{JEL code}: C21, J24, J31, I26
\end{abstract}

\section{Introduction}
Assessing the quality of matching and the degree of affinity between two entities is a widely adopted practice in empirical research. Affinity is typically dissected into observed and unobserved components. Observed affinity is often assessed by estimating the coefficient of the interaction term that incorporates the observable characteristics of the two entities. On the other hand, unobserved affinity is estimated using match fixed effects, which are captured by the coefficient of the interaction term involving dummy variables corresponding to the entities.
This methodology is prevalent in labor and education economics, where it plays a crucial role in managing and interpreting match quality across various pairs, such as teacher-student, worker-company, and worker-job relationships. Our study focuses on clarifying the process of identifying match fixed effects.

As an illustrative example, \cite{inoue2023teachers} investigate the impact of a teacher's major on students' achievement using the following econometric model:
\begin{align*}
    Y_{ifj} = \beta Major_{fj} + \delta_f + \eta_{ij} + \varepsilon_{ifj},
\end{align*}
where \( Y_{ifj} \) represents the science test score of student \( i \) in subfield \( f \) within class \( j \). The parameter \( \beta \) denotes the coefficient of interest, and \( Major_{fj} \) is an indicator variable that indicates whether the teacher's major field in natural science matches the subfield of the student's test score. \( \delta_{f} \) denotes the fixed effect specific to subfield \( f \), while \( \eta_{ij} \) represents the student-teacher fixed effects, accounting for any subfield-invariant determinants of science test scores between student \( i \) and teacher \( j \), thereby capturing their unobserved affinity. The authors find a significant increase in R-squared upon introducing student-teacher fixed effects, underscoring the importance of match fixed effects in their analysis.

Similar methodologies have been employed in other studies, such as examining the pitcher-catcher fixed effect on strikeout likelihood \citep{biolsi2022task}, worker-company fixed effects on worker income \citep{woodcock2015match, mittag2019simple}, worker-company fixed effects on worker turnover rates \citep{ferreira2011measuring}, student-school fixed effects on student scores \citep{ovidi2022parents}, teacher-school fixed effects on student test scores \citep{jackson2013match}, and student-university fixed effects on worker income post-graduation \citep{dillon2020consequences}. 
%Additionally, studies have also controlled for two-way fixed effects, separately considering fixed effects on one side and the other.\textcolor{blue}{Check TWFE in panel model.}

The interpretation of reported parameters of match fixed effects as indicators of affinity for comparison with all matches remains ambiguous, despite their widespread estimation. To address this issue, our paper centers on a standard match fixed effect model and distinguishes between relative and absolute match fixed effects. We contend that absolute match fixed effects lack identification in the absence of specific restrictions, whereas relative match fixed effects are identifiable.
Subsequently, we introduce location normalization conditions that make identification and interpretation of absolute match fixed effect parameters possible. These conditions validate comparisons of unobserved match quality between individuals, thereby enhancing the understanding of unobserved affinity in empirical research contexts.



\section{Model}
We consider the following typical setting.
Suppose that we can observe $I$ workers indexed by $i$ and $J$ job categories indexed by $j$ at time $t=1,\cdots,T$.
Worker $i$ makes some effort to work on a task categorized in job category $j$ at time $t$.
For avoiding later notational complexity, we do not include time fixed effects like a panel regression, but the inclusion does not affect our findings.
We consider the following regression model:
\begin{align}
Y_{ijt}&=\alpha_{i}+\beta_{j}+\mu_{ij}+ X_{ijt}'\gamma+\varepsilon_{ijt}, \label{eq:orignal_regression}\\
&=\sum_{i'=1}^{I}\alpha_{i'}1(i=i')+\sum_{j'=1}^{J}\beta_{j'}1(j=j')+\sum_{i'=1}^{I}\sum_{j'=1}^{J}\mu_{i'j'}1(i=i',j=j')+ X_{ijt}'\gamma+\varepsilon_{ijt}\label{eq:matrix_regression}
\end{align}
where $Y_{ijt}$ is the outcome of worker $i$ for task $j$ at time $t$, which is measured by the time spent on the effort, $X_{ij}$ is $d$-dimensional covariates consisting of observed characteristics of worker $i$ and job category $j$ and its interaction at time $t$, $\alpha_{i}$ is worker $i$'s fixed effect, $\beta_{j}$ is job category $j$'s fixed effect, and $\mu_{ij}$ is $(i,j)$-match fixed effect which is of our interest, $\gamma$ is a $d$-dimensional vector of parameters, $1(\cdot)$ is an indicator function, and $\varepsilon_{ijt}$ is an error term assumed to be drawn i.i.d from some known distribution. 
Note that we explicitly decompose affinity between worker $i$ and job category $j$ into two parts, that is, $X_{ijt}'\gamma$ and $\mu_{ij}$ as the observed and unobserved affinities.
For later discussion, we call $\mu_{ij}$ the \textit{absolute} match effect of worker $i$ and job category $j$. Similarly, we call $\alpha_{i}$ and $\beta_{j}$ the absolute fixed effects.\footnote{Using Equation \eqref{eq:matrix_regression}, matrix representation is described as $Y=X\gamma+1_{I}\alpha+1_{J}\beta+1_{IJ}\mu+\varepsilon=X\gamma+\tilde{X}\delta + \varepsilon$
where $Y$ is $IJT\times 1$, $X$ is $IJT\times d$, $\gamma$ is $d\times 1$, $1_{I}$ is $IJT\times I$, $\alpha$ is $I\times 1$, $1_{J}$ is $IJT\times J$, $\beta$ is $J\times 1$, $1_{IJ}$ is $IJT\times IJ$, $\mu$ is $IJ\times 1$, $\varepsilon$ is $IJT\times 1$, and denote $\tilde{X}=[1_{I} 1_{J} 1_{IJ}]$ and $\delta=[\alpha^T \beta^T \mu^T]^T$. Then, define $I$ as $IJT\times 1$ one vector and $M=I-\tilde{X}(\tilde{X}^T\tilde{X})^{-1}\tilde{X}^{T}$ as the annihilator matrix for $\tilde{X}$. The OLS estimator of $\delta$ is obtained as $\hat{\delta}=(\tilde{X}^T M\tilde{X})^{-1}(\tilde{X}^T MY)$, so the standard full rank condition for $\delta$ is $Rank(\tilde{X}^T M\tilde{X})=(I+J+IJ)$. See \cite{hansen2022econometrics} Chapter 3.16 for reference. However, the condition does not hold due to multicollinearity.}

In the context of the standard fixed effect model, when there are \( I \) groups, typically \( I - 1 \) fixed effects are incorporated, alongside a constant term, to avoid multicollinearity that would arise from including the \( I \)-th group. The match fixed effect case is more complex. Similarly, for a worker \( i \neq 1 \) and job category \( j \neq 1 \), Equation \eqref{eq:orignal_regression} can be rewritten as
\begin{align}
    Y_{ijt}&=\underbrace{\alpha_{1}+\beta_{1}+\mu_{11}}_{\text{constant}}\nonumber\\
    &+\underbrace{(\beta_{j}-\beta_{1})+(\mu_{1j}-\mu_{11})}_{\text{job category }j\text{'s relative fixed effect}}\nonumber\\
    &+\underbrace{(\alpha_{i}-\alpha_{1})+(\mu_{i1}-\mu_{11})}_{\text{worker }i\text{'s relative fixed effect}}\nonumber\\
    &+\underbrace{(\mu_{ij}+\mu_{11}-\mu_{1j}-\mu_{i1})}_{(i,j)\text{'s relative match effect}}+ X_{ijt}'\gamma+\varepsilon_{ijt},
\end{align}
where the first line is a constant parameter normalized to worker 1 and job category 1 which are arbitrarily chosen, the second line is called job category $j$'s \textit{relative} fixed effect which is the fixed effect relative to job category 1's fixed effects, the third line is called worker $i$'s \textit{relative} fixed effect which is the fixed effect relative to worker 1's fixed effects, the third line is called the \textit{relative} match effect of worker $i$ and job category $j$ relative to worker 1 and job category 1.
Avoiding multicollinearity, we can identify and estimate these relative fixed effects and $\gamma$ instead of absolute fixed effects. 
Statistical software automatically reports the estimates of the relative fixed effects.\footnote{Author's github page provides Monte Carlo simulation code and results for investigating finite sample performance on estimating relative fixed effects.}

The direct interpretation of relative fixed effects is that they indicate how well a match compares to a specific reference match, rather than categorizing a match as inherently "good" or "bad" in comparison to all other matches. When researchers aim to simply control for match fixed effects, using relative match fixed effects suffices. However, if the researcher's focus is solely on the overall affinity without distinguishing between observed and unobserved components, the predicted value of \( Y_{ijt} \) derived from relative fixed effects serves adequately.
On the other hand, if the goal is to measure and understand unobserved affinity, such as the quality of personality match, relative match fixed effects become meaningless. In such cases, obtaining absolute fixed effects becomes essential for meaningful comparison and interpretation of unobserved affinity.

Our central question posed is: ``Can we derive the absolute fixed effects from the estimated relative fixed effects without imposing any restrictions?" In mathematical terms, ``Can we solve the system of equations involving relative fixed effects for absolute fixed effects without restrictions?"
Our conclusion is in the negative: No, we cannot achieve this without imposing constraints.
Subsequently, we propose location normalization restrictions that are necessary for identifying the absolute fixed effects \( \mu_{ij} \). 
The results are summarized in Proposition \ref{prop:main_results}.
\begin{proposition}\label{prop:main_results}
For regression model \eqref{eq:orignal_regression}, the following results hold.
\begin{enumerate}
    \item Without any restriction, the absolute match effect $\mu_{ij}$ is not identified. 
    \item With $\sum_i \mu_{i j}=\sum_j \mu_{i j}=0$, the absolute match effect $\mu_{ij}$ for all $i$ and $j$ is identified by relative fixed effects. 
    \item With restriction $\sum_i \mu_{i j}=\sum_j \mu_{i j}=\sum_i \alpha_i=\sum_j \beta_j=0$ for all $i$ and $j$, the absolute fixed effects $\alpha_i$, $\beta_j$, and absolute match effect $\mu_{ij}$ for all $i$ and $j$ are identified by relative fixed effects.
\end{enumerate}

\end{proposition}
See the proof in Appendix \ref{sec:proof}.
Intuitively, the conditions outlined are effective for location normalization. The restricted match effect indicates how much better the match is compared to the average match, rather than a specific match. Consequently, it yields positive values when the match is relatively superior and negative values when it is relatively inferior compared to the average match. This approach ensures interpretability across workers and categories, facilitating meaningful comparisons of unobserved affinity.

Consider an illustrative example with $I=2$ and $J=3$. The relative match effect is estimated as
\begin{align*}
    \mu'_1 = \mu_{11}+\mu_{22}-\mu_{12}-\mu_{21} \\
    \mu'_2 = \mu_{11} +\mu_{23} -\mu_{13} -\mu_{21}
\end{align*}
and all the other relative match effect is calculated from $\mu'_1$ and $\mu'_2$. Without zero sum restrictions, absolute match effect $\mu_{ij}$ is not identifiable, but with restrictions $\sum_i\mu_{ij}=\sum_j\mu_{ij}=0$, the match effect is calculated as\footnote{You can derive $\mu_{11}$ from $\mu'_1+\mu'_2=2\mu_{11}+\mu_{22}+\mu_{23}-\mu_{12}-\mu_{13}-2\mu_{21}=6\mu_{11}$ by using the restrictions. And $\mu_{11}=-\mu_{21}=(\mu'_1+\mu'_2)/6$. Then, from the equations, you get $\mu_{12}=-\mu_{21}=(2\mu_{11}-\mu'_1)/2$ and $\mu_{13}=-\mu_{23}=(2\mu_{11}-\mu'_2)/2$. }
\begin{align*}
     \mu_{11} &= \frac{\mu'_{1}}{6} + \frac{\mu'_{2}}{6}\\
     \mu_{12} &= - \frac{\mu'_{1}}{3} + \frac{\mu'_{2}}{6}\\
     \mu_{13} &= \frac{\mu'_{1}}{6} - \frac{\mu'_{2}}{3}\\
     \mu_{21} &= - \frac{\mu'_{1}}{6} - \frac{\mu'_{2}}{6}\\
     \mu_{22} &= \frac{\mu'_{1}}{3} - \frac{\mu'_{2}}{6}\\
     \mu_{23} &= - \frac{\mu'_{1}}{6} + \frac{\mu'_{2}}{3}
\end{align*}

%\section{Empirical exercise}

%\textcolor{blue}{As in \cite{inoue2023teachers}, we use data of middle school students, taken from the Trends in International Mathematics and Science Study (TIMSS) for 2007. Using the same specification as \cite{inoue2023teachers}, we estimate relative match fixed effects without location normalization and absolute match fixed effects with location normalization. Then, we compare these distributions or only provides the distribution of absolute match fixed effects. [TBA]}






\section{Conclusion}
We investigate the estimation of affinities through match fixed effects, decomposing them into observed and unobserved components. It is emphasized that without appropriate restrictions, we can only estimate relative fixed effects—fixed effects relative to normalized values—that lack interpretability as affinities. To achieve meaningful interpretations of absolute match effects, we introduce theoretical constraints on parameters.
%\textcolor{blue}{Using data of middle school students, taken from the Trends in International Mathematics and Science Study (TIMSS) for 2007, we provide the distribution of absolute match fixed effects.}

\paragraph{Acknowledgments}
We thank Shunya Noda and Shosei Sakaguchi for their valuable advice. This work was supported by JST ERATO Grant Number JPMJER2301, Japan.


\bibliographystyle{ecca}
\bibliography{match_effect}



\appendix
\section{Proof (Online appendix)}\label{sec:proof}
\begin{proof}
    Before showing the proof, we provide the overview. The system of linear equations about parameters to be solved without any restrictions is
    \begin{equation*}
        \begin{pmatrix}
            A \\ B \\ C
        \end{pmatrix}
        \begin{pmatrix}
            \mathbf{\alpha} \\ \mathbf{\beta} \\ \mathbf{\mu}
        \end{pmatrix}=
        \begin{pmatrix}
            \mathbf{\alpha'} \\ \mathbf{\beta'} \\ \mathbf{\mu'}
        \end{pmatrix}
    \end{equation*}
    where $A, B$ and $C$ are the coefficient matrix for relative effects and $\alpha',\beta'$ and $\mu'$ are calculated relative fixed effects defined later.
    Let $T$ denote transpose.
    Then if $rank\begin{pmatrix}
        [A^T&B^T&C^T]^T
    \end{pmatrix}=rank\begin{pmatrix}
        [\alpha^T&\beta^T&\mu^T]^T
    \end{pmatrix}=I+J+IJ$, the absolute fixed effects and match effects are just identified. Thus, we will check the rank conditions to determine whether a system of linear equations is underdetermined (i.e., underidentified), meaning there are fewer equations than unknowns.\\
    Since there are $I$ workers and $J$ categories, there are $I+J$ unknown absolute fixed effect parameters and $IJ$ unknown absolute match effect parameters. The fixed effects of $(i,j)$ relative to $(i_0,j_0)$, $\alpha'_{i_0j_0,i},\beta'_{i_0j_0,j},\mu'_{i_0j_0,ij}$ is calculated by 
    \begin{align}
        \alpha'_{i_0j_0,i} &= \alpha_i-\alpha_{i_0}+\mu_{ij_0}-\mu_{i_0j_0}\label{eq:relative_alpha} \\
        \beta'_{i_0j_0,j} &= \beta_j-\beta_{j_0}+\mu_{i_0j}-\mu_{i_0j_0} \label{eq:relative_beta} \\
        \mu'_{i_0j_0,ij} &= \mu_{ij}+\mu_{i_0j_0}-\mu_{ij_0}-\mu_{i_0j}. \label{eq:relative_mu} 
    \end{align}
    First, we prove that absolute match effects are not identifiable without any restrictions.
    Let coefficient matrices $A$, $B$ and $C$ represent coefficient of $\alpha_i$, $\beta_j$ and $\mu_{ij}$ in Equations \eqref{eq:relative_alpha}, \eqref{eq:relative_beta}, and \eqref{eq:relative_mu} respectively as follows:
    \arraycolsep=4pt
    \begin{equation*}
        \begin{array}{crcccccccccccccl}
            \multirow{4}{*}{A\ =}&\ldelim({4}{14pt}[] & a^{11,2}_{1} & \cdots & a^{11,2}_{I} & 0 & \cdots & 0 & c^{11,2}_{11} & \cdots & c^1_{1J} & \cdots & c^{11,2}_{I1} & \cdots & c^{11,2}_{IJ} & \rdelim){4}{14pt}[] \\
            && a^{11,3}_{1} & \cdots & a^{11,3}_{I} & 0 & \cdots & 0 &c^{11,3}_{11} & \cdots & c^{11,3}_{1J} & \cdots & c^{11,3}_{I1} & \cdots & c^{11,3}_{IJ} & \\
            && \vdots & \vdots & \vdots & \vdots & \vdots & \vdots & \vdots & \vdots &\vdots & \vdots & \vdots & \vdots & \vdots &\\
            && a^{IJ,I-1}_{1} & \cdots & a^{IJ,I-1}_{I} & 0 & \cdots & 0 &c^{IJ,I-1}_{11} & \cdots & c^{IJ,I-1}_{1J} & \cdots & c^{IJ,I-1}_{I1} & \cdots & c^{IJ,I-1}_{IJ} &\\
            \multirow{4}{*}{B\ =}&\ldelim({4}{14pt}[] & 0 & \cdots & 0 & b^{11,2}_{1} & \cdots & b^{11,2}_{J} & c^{11,2}_{11} & \cdots & c^{11,2}_{1J}  & \cdots & c^{11,2}_{I1} & \cdots & c^{11,2}_{IJ} & \rdelim){4}{14pt}[] \\
            && 0 & \cdots & 0 & b^{11,3}_{1} & \cdots & b^{11,3}_{J} &c^{11,3}_{11} & \cdots & c^{11,3}_{1J}  & \cdots & c^{11,3}_{I1} & \cdots & c^{11,3}_{IJ} & \\
            && \vdots & \vdots & \vdots & \vdots & \vdots & \vdots & \vdots & \vdots & \vdots &\vdots & \vdots & \vdots & \vdots &\\
            && 0 & \cdots & 0 & b^{IJ,J-1}_{1} & \cdots & b^{IJ,J-1}_{J} &c^{IJ,J-1}_{11} & \cdots & c^{IJ,J-1}_{1J}  & \cdots & c^{IJ,J-1}_{I1} & \cdots & c^{IJ,J-1}_{IJ} &\\
            \multirow{4}{*}{C\ =}&\ldelim({4}{14pt}[] & 0 & \cdots & 0 & 0 & \cdots & 0 & c^{11,2}_{11} & \cdots & c^{11,2}_{1J}  & \cdots & c^{11,2}_{I1} & \cdots & c^{11,2}_{IJ} & \rdelim){4}{14pt}[] \\
            && 0 & \cdots & 0 & 0 & \cdots & 0 &c^{11,3}_{11} & \cdots & c^{11,3}_{1J}  & \cdots & c^{11,3}_{I1} & \cdots & c^{11,3}_{IJ} & \\
            && \vdots & \vdots & \vdots & \vdots & \vdots & \vdots & \vdots & \vdots & \vdots &\vdots & \vdots & \vdots & \vdots &\\
            && 0 & \cdots & 0 & 0 & \cdots & 0 &c^{IJ,J-1}_{11} & \cdots & c^{IJ,J-1}_{1J}  & \cdots & c^{IJ,J-1}_{I1} & \cdots & c^{IJ,J-1}_{IJ} &\\
             && \multicolumn{13}{l}{\underbrace{\hspace{7.5em}}_{I}\hspace{1.5em}\underbrace{\hspace{7.5em}}_{J}\hspace{1.5em}\underbrace{\hspace{18em}}_{IJ}} & \\
        \end{array}
    \end{equation*}
    \arraycolsep=5pt
    where $a^{i_0j_0,i}_{k}$ and $b^{j_0j_0,i}_{k}$ represent the coefficients of $\alpha_k$ and $\beta_k$, and $c^{i_0j_0,i}_{kl}$ represents the coefficient of $\mu_{kl}$ in the equation $\mu'_{i_0j_0,ij}=\alpha_i-\alpha_{i_0}+\mu_{ij_0}-\mu_{i_0j_0}$. 
    Every element in each matrix is either $0, 1$ or $-1$.
    
    First, consider the matrix $C$. 
    Assume $c^{i_0j_0,ij}_{kl}=1\ (k\neq i_0,\ l\neq j_0)$, then $c^{i_0j_0,ij}_{i_0j_0}=1$, $c^{i_0j_0,ij}_{i_0l}=c^{i_0j_0,ij}_{kj_0}=-1$ and $c^{i_0j_0,ij}_{k'l'}=0$ for $k'\notin \{k,i_0\},\ l'\notin \{l,j_0\}$. 
    Then, $c^{ij,i_0j_0}=c^{i_0j_0,ij}$, $c^{ij_0,i_0j}=-c^{i_0j_0,ij}$, and $c^{i_0'j_0',ij}=c^{i_0j_0,ij}+c^{i_0j_0,i_0'j_0'}-c^{i_0j_0,i_0'j}-c^{i_0j_0,ij_0'}$ hold. 
    Here fix $i_0=j_0=1$. 
    Then $c^{i_0'j_0',i'j'}$ can be obtained from $c^{11,ij}$. 
    Thus, $rank(C)=(I-1)(J-1)$.
    Similarly, because $a^{i_0j_0,i}=a^{ij_0,i_0}$ and $a^{i_0'j_0',i}=a^{11,i}-a^{11,i_0'}+c^{ij_0',i_0'1}$, $rank
    \begin{pmatrix}
        [A^T& C^T]^T
    \end{pmatrix} = 
    (I-1)+(I-1)(J-1)=(I-1)J$.
    Also, because $b^{i_0j_0,j}=b^{i_0j,j_0}$ and $b^{i_0'j_0',j}=b^{11,j}-b^{11,j_0'}+c^{i_0'j,1j_0'}$, $rank\begin{pmatrix}
        [A^T & B^T & C^T]^T
    \end{pmatrix}= (J-1)+(I-1)J=IJ-1$.
    
    This concludes that $rank\begin{pmatrix}         [A^T & B^T & C^T]^T     \end{pmatrix}=IJ-1<IJ$ and absolute match effects are not identifiable. 
    Note that without match fixed effect, then all elements in $C$ is $0$, $rank\begin{pmatrix}
        [A^T & B^T]^T
    \end{pmatrix}=I+J-2$, and thus $\alpha$ and $\beta$ is not identifiable without additional restrictions. 
    
    Next, we prove that absolute match effects are identifiable with restrictions $\sum_i\mu_{ij}=\sum_j\mu_{ij}=1$.
    Each restriction is independent from every equation above and the number of independent restrictions is $I+J-1$. 
    Therefore, $rank\begin{pmatrix}
        [C^T & R_1^T]^T
    \end{pmatrix}=IJ$ where $R_1$ is the coefficient matrix for the restrictions $\sum_i\mu_{ij}=\sum_j\mu_{ij}=1$ as
    \begin{equation*}
        R_1 = 
        \begin{array}{rccccccccccccccccll}
            & \multicolumn{16}{l}{\hspace{10em}\overbrace{\hspace{16em}}^{IJ}}&&\\
            \ldelim({7}{14pt}[] & 0 & \cdots & 0 & 0 & \cdots & 0 & 1 & \cdots & 1 & 0 & \cdots & 0 & \cdots & 0 & \cdots & 0 & \rdelim){7}{14pt}[]&\rdelim\}{4}{14pt}[I]\\
            & 0 & \cdots & 0 & 0 & \cdots & 0 & 0 & \cdots & 0 & 1 & \cdots & 1 & \cdots & 0 & \cdots & 0 & & \\
            & \vdots & \vdots & \vdots & \vdots & \vdots & \vdots & \vdots & \vdots & \vdots & \vdots & \vdots & \vdots & \vdots & \vdots & \vdots & \vdots & &\\
             & 0 & \cdots & 0 & 0 & \cdots & 0 & 0 & \cdots & 0 & 0 & \cdots & 0 & \cdots & 1 & \cdots & 1 & &\\
              & 0 & \cdots & 0 & 0 & \cdots & 0 & 1 & \cdots & 0 & 1 & \cdots & 0 & \cdots & 1 & \cdots & 0 & & \rdelim\}{3}{14pt}[J]\\
              & \vdots & \vdots & \vdots & \vdots & \vdots & \vdots & \vdots & \vdots & \vdots & \vdots & \vdots & \vdots & \vdots & \vdots & \vdots & \vdots & &\\
              & 0 & \cdots & 0 & 0 & \cdots & 0 & 0 & \cdots & 1 & 0 & \cdots & 1 & \cdots & 0 & \cdots & 1 & & \\
              &\multicolumn{16}{l}{\underbrace{\hspace{4em}}_{I}\ \ \underbrace{\hspace{4em}}_{J}\ \ \underbrace{\hspace{4em}}_{I}\ \ \underbrace{\hspace{4em}}_{I}\hspace{3.5em} \underbrace{\hspace{4em}}_{I}} &&
        \end{array}.
    \end{equation*}
    Then the absolute match effect can be identified since the rank is the same as the number of absolute match effect parameters.

    Finally, we prove that absolute match effect and fixed effects are identifiable with restrictions $\sum_i \mu_{i j}=\sum_j \mu_{i j}=\sum_i \alpha_i=\sum_j \beta_j=0$.
    Each row in the coefficient matrix for restrictions $\sum_i \alpha_i=\sum_j \beta_j=0$, $R_2$ and $R_3$ defined as
    \begin{equation*}
        \begin{array}{cc}
             R_2=& \begin{pmatrix}
                 1 & \cdots & 1 & 0 & \cdots & 0 & 0 & \cdots & 0
             \end{pmatrix}\\
             R_3=& \begin{pmatrix}
                 0 & \cdots & 0 & 1 & \cdots & 1 & 0 & \cdots & 0
             \end{pmatrix} \\
             & \multicolumn{1}{l}{\hspace{1em}\underbrace{\hspace{3.5em}}_{I}\hspace{1.2em}\underbrace{\hspace{3.5em}}_{J}\hspace{1.2em}\underbrace{\hspace{3.5em}}_{IJ}} 
        \end{array}
    \end{equation*}
    are independent from every row in $A,B,C$ and $R_1$ and they have rank of $1$. 
    This is because the number of $a$'s in a row which is not zero is 2 while at least $I$ elements remains not zero by adding the rows in $R_2$, and similar for $R_3$ independence.
    Therefore, $rank\begin{pmatrix}
        [A^T & B^T & C^T & R_1^T & R_2^T & R_3^T]^T
    \end{pmatrix}=IJ+I+J$ which equals to the number of parameters, and thus absolute match effect and fixed effect parameters are identifiable.
\end{proof}

\end{document}









